%%%%%%%%%%%%%%%%%%%%%%%%%%%%%%%%%%%%%%%%%%%%%%%%%%%%%%%%%%%%%%%%%%%%%%%%%%%%%%%%
% File: RHITBeamer
% Description: Template for creating presentations using the RHITslides theme.
%
% Date:
% Modified:
%
% Notes:
%   1. Created by Eric Reyes (Rose-Hulman Insitute of Technology).

%%%%%%%%%%%%%%%%%%%%%%%%%%%%%%%%%%%%%%%%%%%%%%%%%%%%%%%%%%%%%%%%%%%%%%%%%%%%%%%%
% Class Definition
%   The beamer class is extremely flexible and useful for creating Latex 
%   presentations. While other presentation classes exist, this one is easy to
%   learn and very powerful.
%
%   The color scheme and style are part of the beamer theme RHITslides, which
%   adheres to the university style guide. There are some things that are in 
%   this template unique to this style file which are highlighted as we go.
%
% Options for RHITslides
%   simpletoc: If specified, the table of contents will only contain the section
%     names and omit the subsection and subsubsection names specified in the
%     document.
%   headline=empty: The headline is an empty box of a single color at the top
%     of every slide (default).
%   headline=navigation: A horizontal navigation consisting of the sections is
%     listed in a single line. Note that subsections are not given. Since this
%     appears in a single line, section names should be short and no more than
%     four or five are recommended.
%   headline=sectionsubsection: On the left, the current section is given, and
%     on the right, the current subsection.
%   headline=titlesection: On the left, the title of the talk is given in 
%     italics, on the right, the current section.
%   footline=empty: The footline is an empty box of a single color at the bottom
%     of every slide (default).
%   footline=logoauthordate: On the left, the logo is placed, in the center, the
%     author, on the right the date and page number.
%   footline=logodate: Same as logoauthordate, but no author is shown.
%   footline=logoonly: Same as logodate, but not date or page number is given.
%   footline=pageonly: In the far right corner, the page number is shown 
%     (default).
%   sidebar=navigation: No left sidebar. The right sidebar contains basic 
%     navigation symbols (default).
%   sidebar=none: No left or right sidebar.

\documentclass{beamer}
\usetheme[simpletoc,
				 headline=titlesection,
				 footline=logoauthordate,
				 sidebar=navigation]{RHITslides}



%%%%%%%%%%%%%%%%%%%%%%%%%%%%%%%%%%%%%%%%%%%%%%%%%%%%%%%%%%%%%%%%%%%%%%%%%%%%%%%%
% Load Packages
%   These additional packages control the layout and ability to use specific
%   features. Include additional packages as needed.
%
%   1. amsmath,amsthm,amssymb:
%        Allow the use of AMS mathematical symbols, theorem environments, and
%        math commands.
%   2. booktabs
%        Tables have improved spacing for a more professional look.
%   3. graphicx
%        Allows more flexibility for importing graphics.
\usepackage{amsmath,amsthm,amssymb,booktabs,graphicx}



%%%%%%%%%%%%%%%%%%%%%%%%%%%%%%%%%%%%%%%%%%%%%%%%%%%%%%%%%%%%%%%%%%%%%%%%%%%%%%%%
% Set Document Information
%   These pieces are used to create the title page slide and logo. Shorter
%   versions (specified optionally) are used in the headers and footers.
%
%   The title graphic and logo are the official RHIT logo. In order to use it 
%   properly, only the height or width should be set (but not both). Using
%   two different ways of referring to the graphic allows for it to be resized
%   on the front page. Also, to ensure proper placement in the footer, only
%   totalheight should be specified and should be less than 4.75ex (the total
%   height of the footer).
\title[short title]{Official Title}
\subtitle{the subtitle, if applicable}
\author[short author]{Full Author Name}
\institute[RHIT]{}
\date[short date]{Official Location and Date}

\titlegraphic{\includegraphics[width=0.5\textwidth]{RHIT_logo.png}}
\logo{\includegraphics[totalheight=2.75ex]{RHIT_logo.png}}



%%%%%%%%%%%%%%%%%%%%%%%%%%%%%%%%%%%%%%%%%%%%%%%%%%%%%%%%%%%%%%%%%%%%%%%%%%%%%%%%
% Title Page and Outline
%   You could specify \titlepage and \tableofcontents as in any other document.
%   However, the RHITslides theme also provides two unique commands: \titlepg
%   and \tocpg. These commands create the title page and table of contents page
%   such that the header and footer do not contain all that information, which
%   is unnecessary on these first two pages. In addition, these pages will not
%   be numbered, and slide (1) will be the first slide following the table of
%   contents. \outline takes one required parameter, the title for the page.
%   If left empty, no title line is created. To get a blank title line, specify
%   something like $\phantom{X}$.
\begin{document}

\titlepg

\tocpg{Outline}



%%%%%%%%%%%%%%%%%%%%%%%%%%%%%%%%%%%%%%%%%%%%%%%%%%%%%%%%%%%%%%%%%%%%%%%%%%%%%%%%
% Section I 
%   A new section can be started using the \section command as usual. However,
%   the RHITslides theme also proves the command \sectionpg. This command
%   defines a new section AND creates a page to introduce the section. This is
%   useful for making transitions in longer talks. The command takes one
%   required argument, the section name (which will appear in the table of
%   contents and in the header if specified). It also takes one optional
%   argument, a longer version of the section name which will appear on the 
%   introduction slide.
\sectionpg[Longer Section I Name]{Section I Name}

% A basic list
\begin{frame}
  \frametitle{A list}
  \begin{itemize}
    \item Items denoted by red balls.
    \item Enumerate and Description environments similar.
  \end{itemize}
\end{frame}

% A Table
\begin{frame}
	\frametitle{A Table}
	\begin{center}
		\begin{tabular}{lrr}\toprule
			Column 1 & Column 2 & Column 3 \\ \midrule
			entry 1 & entry 2 & entry 3 \\ \bottomrule
		\end{tabular}
	\end{center}
\end{frame}



%%%%%%%%%%%%%%%%%%%%%%%%%%%%%%%%%%%%%%%%%%%%%%%%%%%%%%%%%%%%%%%%%%%%%%%%%%%%%%%%
% Section II
\sectionpg[Longer Section II Name]{Section II Name}

% A Block
\begin{frame}
  \frametitle{A Basic Block}
  \begin{block}{Block Color Scheme}
    Header is white text on red background, and the block body is a light grey
    background with black text.
  \end{block}
  
  {\usebeamercolor[fg]{structure} Specific words} can also be highlighted 
  in this color using ``usebeamercolor'' command.
\end{frame}

% An Example
\begin{frame}
	\frametitle{An Example}
	\begin{example}
		The example block places the name ``Example'' in the header on a dark grey
		background.  The block body is a light grey with black text.
	\end{example}
	
	{\usebeamercolor[fg]{example text} Specific words} can also be highlighted for 
	examples using ``usebeamercolor'' command.
\end{frame}

% An Alert
\begin{frame}
	\frametitle{An Alert}
	\begin{alertblock}{Alert Color Scheme}
		Header is white text on dark blue blackground, and the block body is a light
		version of this blue with black text.
	\end{alertblock}
	
	{\usebeamercolor[fg]{alerted text} Specific words} can also be highlighted for 
	alerts using ``usebeamercolor'' command.
\end{frame}

% A Theorem
\begin{frame}
	\frametitle{A Theorem}
	\begin{theorem}
		Text goes here...
	\end{theorem}
	
	\begin{proof}
		Text goes here...
	\end{proof}
\end{frame}


% This is just to highlight the effect of subsections on the presentation.
\subsection{Subsection Title}
\begin{frame}
  \frametitle{Using References}
  
  Here are some references:~\cite{Example1ref} and~\cite{Example2ref}.
\end{frame}

 
 

%%%%%%%%%%%%%%%%%%%%%%%%%%%%%%%%%%%%%%%%%%%%%%%%%%%%%%%%%%%%%%%%%%%%%%%%%%%%%%%%
% References
%   I don't use a new section for references because I don't want it to appear
%   in the Outline. But, if you want it there, use \section or \sectionpg as
%   usual.
%
%   Packages like natbib don't work with beamer. So, I just define references
%   the hard way. Start it with \begin{thebibliography}{A} where A is the name
%   of the bibitem with the longest title. 
%
%   Each bibitem has the form [how it looks in text]{ref name} citation text.
%
%   You can also include this in the appendix section, as I do here. Appendices
%   do not appear in the outline.
\appendix

\section{References}
\begin{frame}
	\frametitle{References}
	\begin{thebibliography}{Example 1 (2010)}
	
	\bibitem[Example 1 (2010)]{Example1ref}
		AuthorLast~AuthorFirst.
		\newblock Title goes here.
		\newblock{\em Journal Name Here}, Vol:pages-pages, year.
		
	\bibitem[Example 2 (2010)]{Example2ref}
		AuthorLast~AuthorFirst.
		\newblock Title goes here.
		\newblock{\em Journal Name Here}, Vol:pages-pages, year.
	\end{thebibliography}
\end{frame}

\section{Appendix A}
\begin{frame}
	\frametitle{Slide in the Appendix}
	
	This slide is in Appendix A.
\end{frame}

\end{document}
