%%%%%%%%%%%%%%%%%%%%%%%%%%%%%%%%%%%%%%%%%%%%%%%%%%%%%%%%%%%%%%%%%%%%%%%%%%%%%%%%
% File: RHITposter.tex
% Description: Template for creating a poster with the RHIT color theme.
%
% Date:
% Modified:
%
% Notes:
%   1. Created by Eric Reyes (Rose-Hulman Insitute of Technology).

%%%%%%%%%%%%%%%%%%%%%%%%%%%%%%%%%%%%%%%%%%%%%%%%%%%%%%%%%%%%%%%%%%%%%%%%%%%%%%%%
% Class Definition
%   The RHITposter is based on the beamer presentation document class due to
%   its extreme flexibility. And, you will be familiar with this class if you
%   do your presentations in beamer. The t option is set to align everything at
%   to top of the slide.
%
%   Packages:
%     beamerposter: This package sets up beamer to handle a poster. You can
%       customize the size and scaling, etc. See its documentation for more
%       details. Here, we set up a horizontal poster that is typical size for
%       printing.
%     RHITposter: Loads the RHIT color scheme and specifications. The buffer
%       option places a buffer below the footline so that you can make items
%       flush at the top and bottom of page.
%     amsmath,amssymb,graphicx,booktabs: These packages load AMS math tools,
%       graphic options, and nice tables.  Add and take away from this list
%       as you see fit.

\documentclass[t,12pt]{beamer}
\usetheme[buffer]{RHITposter}
\usepackage[orientation=landscape,size=custom,scale=1.2,width=93,height=52]{beamerposter}
\usepackage{amsmath,amssymb,booktabs,graphicx}





%%%%%%%%%%%%%%%%%%%%%%%%%%%%%%%%%%%%%%%%%%%%%%%%%%%%%%%%%%%%%%%%%%%%%%%%%%%%%%%%
% Additional Information
%   Provides additional information for placement on poster. If you need your
%   title to run 2 lines, use the \subtitle{} command for your second line. We
%   note that the logo cannot extend past 1/3 of the paper width; so, the aspect
%   ratio should be fixed to prevent this.

\title{Title of Presentation}
\author{Full Author 1, Full Author 2}
\institute{Department, University}
\logo{\includegraphics[width=0.15\paperwidth,keepaspectratio]
	{RHIT_logo.png}}
	
\email{username@rose-hulman.edu}
\website{www.rose-hulman.edu/$\sim$username}
\date{Title of Workshop, (year)}



%%%%%%%%%%%%%%%%%%%%%%%%%%%%%%%%%%%%%%%%%%%%%%%%%%%%%%%%%%%%%%%%%%%%%%%%%%%%%%%%
% Construct Poster
%   The poster acts as one frame from a beamer presentation. Therefore, it is
%   easiest to divide the frame into columns and work from there. You can align
%   the columns at the top to keep everything flush.
%
%   Three columns is standard. You can give them each 0.33 of the textwidth or
%   divide them as you see fit.

\begin{document}\begin{frame}\begin{columns}[t]

%%% Column 1
\begin{column}{0.33\textwidth}
  \begin{block}{First Block in Column 1}
    Typically, an introduction would go here.  Note that the length of a 
    column is determined by the length of the blocks in the column.
  \end{block}

  \begin{block}{Second Block in Column 1}
  		More text goes here.  Note that the length of a column would be determined
		by the number of blocks and their length. If the ``buffer'' option is 
		specified in the RHITposter theme, then each column can be flush at the
		bottom by inserting a blank box that has width smaller than the block and
		height equal to the textheight. The block will technically run off the page,
		but with the buffer, it will create a border that looks like it stopped just
		shy of the bottom of the page.
		
		% create blank box
		\begin{beamercolorbox}[wd=1ex,ht=\textheight]{block body}
		\end{beamercolorbox}
	\end{block}
\end{column}


%%% Column 2
\begin{column}{0.33\textwidth}
  \begin{block}{First Block in Column 2}
    This block lines up with the first block in column 1.

  \end{block}
  
  \begin{block}{Second Block in Column 2}
  		Without the blank box, the column would end early making them not flush
		with one another.
	\end{block}
\end{column}


%%% Column 3
\begin{column}{0.33\textwidth}
	\begin{block}{First Block in Column 3}
    Last column. 
  \end{block}

	\begin{block}{Last Block}
		References are typically entered here.  They must be entered by hand, but
		can be referenced as using the ``cite'' command as usual: 
		Reference~\cite{Example1ref} and~\cite{Example2ref}.
		
		\begin{thebibliography}{[1]}
			\bibitem[1]{Example1ref}
			AuthorLast~AuthorFirst.
			\newblock Title goes here.
			\newblock{\em Journal Name Here}, Vol:pages-pages, year.
		
			\bibitem[2]{Example2ref}
			AuthorLast~AuthorFirst.
			\newblock Title goes here.
			\newblock{\em Journal Name Here}, Vol:pages-pages, year.
		\end{thebibliography}
		
		% blank box
		\begin{beamercolorbox}[wd=1ex,ht=\textheight]{block body}
		\end{beamercolorbox}
	\end{block}
\end{column}

\end{columns}\end{frame}\end{document} 